\section*{ВВЕДЕНИЕ}
\addcontentsline{toc}{section}{ВВЕДЕНИЕ}

В современном мире использование изображений растет с каждым днем. В конце концов, большинство людей легче воспринимают любую информацию именно визуально. В то же время, наряду со значительным повышением уровня развития технологий, методы обработки изображений играют очень значительную роль. 
Вот уже около двухсот лет процесс улучшения фотографий продолжается и не прекращается даже сегодня. Новые формы обработки изображений рождаются каждый день, некоторые совершенствуются, другие заменяются более новыми и интересными. Благодаря современным технологиям методы обработки изображений полностью меняют представление об изображении. 

Области применения цифровой обработки изображений весьма разнообразны. Она имеет свое распространение во многих науках, в создании рекламы и т.д.
Если вернуться на несколько лет назад, то можно вспомнить, что фотографии того времени были в основном черно-белыми, лишь постепенно начинавшими приобретать цвет. Но с годами они теряют свое прежнее состояние, теряют свое качество. С совершенствованием технологии удалось добиться того, что вернуть фотографии первоначальный вид -- дело нескольких минут.
Обработка изображений обеспечивает улучшение изображения, сжатие данных для хранения и передачи по каналам связи, а также анализ, распознавание и интерпретацию визуальных изображений.

\textbf{Цель данной работы} -- реализовать программу, способную помочь пользователю обработать выбранное им изображение, а также, наложить на него один или несколько известных фильтров.

Чтобы достичь поставленной цели, требуется решить следующие задачи:

\begin{itemize}[leftmargin=1.6\parindent]
	\item[---] выбрать формат изображения, над которым будут проводиться операции для редактирования изображения;
	\item[---] определить методы фильтрации изображения, рассмотреть их основные характеристики;
	\item[---] познакомиться с известными фильтрами обработки изображения;
	\item[---] полученные знания применить на практике.
\end{itemize}

\pagebreak